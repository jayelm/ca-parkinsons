\documentclass[letterpaper,12pt]{article}
\usepackage{fancyhdr}
\usepackage{amsmath}
\usepackage{amssymb}
\usepackage{bm}
\usepackage{numprint}
\usepackage[margin=1in]{geometry}
\usepackage{graphicx}
% Random packages from
% http://tex.stackexchange.com/questions/50070/landscape-figure-in-latex
% Necessary for sideways pictures
\usepackage{wrapfig}
\usepackage{lscape}
\usepackage{rotating}
\usepackage{epstopdf}
\usepackage{tablefootnote}
% for word wrap verbatim
\usepackage{listings}
\lstset{
   breaklines=true,
   basicstyle=\ttfamily}
% \pagestyle{fancy}
% \lhead{Jesse Mu}
% \rhead{CSCI339 Term Project}
\graphicspath{ {../figures/} }
% Or this, if run from main folder
% \graphicspath{ {./figures/} }



\begin{document}

\title{Cluster Analysis: Identifying Parkinson's Disease Subtypes}
\date{Wednesday, June 10}
\author{Jesse Mu}
\maketitle

\section{Preprocessing}

\subsection{Dataset Description}
951 subjects, 145 metrics, collected 15-4-2012 from Pablo Martinez \'in. Only
19 features used for clustering and/or interpretation.  50 subjects with
missing values of the features to be used in clustering (brought down to 901).
Imputation may be a good idea later on.

\subsection{Selected Features}

Combination of non-motor scale (NMS) symptoms and standard motor symptoms.

\begin{table}[h]
  \centering
  \begin{tabular}{l|l|l|l}
    Name & Type & Format & Description \\
    \hline
    nms\_d1 & byte & \%8.0g & cardiovascular \\
    nms\_d2 & byte & \%8.0g & sleep/fatigue \\
    nms\_d3 & byte & \%8.0g & mood/cognition \\
    nms\_d4 & byte & \%8.0g & percep/hallucinations \\
    nms\_d5 & byte & \%8.0g & attention/memory \\
    nms\_d6 & byte & \%8.0g & gastrointestinal \\
    nms\_d7 & byte & \%8.0g & urinary \\
    nms\_d8 & byte & \%8.0g & sexual function \\
    nms\_d9 & byte & \%8.0g & miscellaneous \\
    tremor & float & \%9.0g & tremor \\
    bradykin & float & \%9.0g & bradykinesia\tablefootnote{Impaired ability to
    adjust the body's position.} \\
    rigidity & float & \%9.0g & rigidity \\
    axial & float & \%9.0g & axial\tablefootnote{Issues affecting the middle of
    the body.} \\
    pigd & float & \%9.0g & postural instability and gait difficulty \\
  \end{tabular}
  \caption{Selected Features and Details}
  \label{tab:selected-features}
\end{table}

\begin{table}[h]
  \centering
  \begin{tabular}{l|l|l|l}
  Name  &       $\mu$ & $\sigma$ & min-max \\
         \hline
nms\_d1&   1.73&  3.35&   0-24 \\
nms\_d2&   8.75&  8.70&   0-48 \\
nms\_d3&   8.68& 11.55&   0-60 \\
nms\_d4&   1.64&  3.86&   0-33 \\
nms\_d5&   5.42&  7.43&   0-36 \\
nms\_d6&   5.53&  6.79&   0-36 \\
nms\_d7&   8.08&  8.94&   0-36 \\
nms\_d8&   3.52&  5.97&   0-24 \\
nms\_d9&   7.13&  7.79&   0-48 \\
tremor&   2.59&  2.58&   0-12 \\
bradykin& 2.40&  1.41&   0-6 \\
rigidity& 2.24&  1.36&   0-6 \\
axial&    3.25&  2.68&   0-12 \\
pigd&     3.31&  2.71&   0-12 \\
  \end{tabular}
  \caption{Descriptive Statistics}
  \label{tab:descriptive-statistics}
\end{table}

\subsection{Dimensionality Reduction: PCA}

May not be useful? If we're trying to identify \emph{clinically} relevant
features, merging them may not be a good idea. Regardless, Figure~\ref{fig:pca}
shows results of preliminary PCA.

\begin{figure}[h]
  \centering
  \includegraphics[width=\linewidth]{pca.pdf}
  \caption{PCA Analysis}
  \label{fig:pca}
\end{figure}

\begin{figure}[h]
  \centering
  \includegraphics[width=\linewidth]{pca-eigenvalues.pdf}
  \caption{Scree test: eigenvalues by factor}
  \label{fig:pca-eigenvalues}
\end{figure}

Figure~\ref{fig:pca-eigenvalues} shows scree test elbow occurs around 2 or 2 or
.4 Also, eigenvalues $1-5 > 1$.

\clearpage
\section{$k$-means}
\subsection{Identifying optimal number of clusters}

\subsubsection{WSS Error Scree Test}

\begin{figure}[h]
  \centering
  \includegraphics[width=\linewidth]{kmeans-wss-error.pdf}
  \caption{Scree test: WSS error by cluster size}
  \label{fig:kmeans-wss-error}
\end{figure}

Figure~\ref{fig:kmeans-wss-error} shows no optimal elbow in scree test! Maybe 2-3?

\subsubsection{Gap Statistic}

Optimal cluster is the local maximum of the gap statistic, but it appears to be
consistently increasing in Figure~\ref{fig:gap-statistic}.

\begin{figure}[h]
  \centering
  \includegraphics[width=\linewidth]{gap-statistic.pdf}
  \caption{Gap statistic by cluster size}
  \label{fig:gap-statistic}
\end{figure}

\subsubsection{Average Silhouette Width}

Figure~\ref{fig:asw} shows average silhouette width as being consistently under
0.25 for all clusters, implying the data is not well structured.

\begin{figure}[h]
  \centering
  \includegraphics[width=\linewidth]{asw.pdf}
  \caption{Average silhouette width by cluster size}
  \label{fig:asw}
\end{figure}

% \subsubsection{\texttt{NbClust} package}

\subsection{Cluster statistics}
% CLUSTERS: 2
% ================================
% Sizes: 229 672
% WithinSS: 6117.986 7695.411
% Sum WithinSS: 13813.4
% CLUSTERS: 3
% ================================
% Sizes: 333 134 434
% WithinSS: 4669.037 4009.387 4153.892
% Sum WithinSS: 12832.32
% CLUSTERS: 4
% ================================
% Sizes: 79 394 275 153
% WithinSS: 2366.585 3356.709 3454.142 2879.508
% Sum WithinSS: 12056.94
\begin{table}[h]
  \centering
  \begin{tabular}{l|l|l|l}
    $k$ & $n$ & Within SS & sum(Within SS) \\
    \hline
    2 & 229/672 & 6118/7695 & 13813 \\
    3 & 333/134/434 & 4669/40009/4154 & 12832 \\
    4 & 79/394/275/153 & 2367/3357/3454/2880  & 12057 \\
  \end{tabular}
  \caption{Cluster statistics}
  \label{tab:cluster-statistics}
\end{table}

\subsection{Silhouette plots}

Available in
Figures~\ref{fig:kmeans-silhouette-2},~\ref{fig:kmeans-silhouette-3},
and~\ref{fig:kmeans-silhouette-4}. Note: constructed with standardized
$z$-score data.

\begin{figure}[h]
  \centering
  \includegraphics[width=\linewidth]{kmeans-silhouette-2.pdf}
  \caption{$k$-means cluster silhouette plot, $k = 2$}
  \label{fig:kmeans-silhouette-2}
\end{figure}

\begin{figure}[h]
  \centering
  \includegraphics[width=\linewidth]{kmeans-silhouette-3.pdf}
  \caption{$k$-means cluster silhouette plot, $k = 3$}
  \label{fig:kmeans-silhouette-3}
\end{figure}

\begin{figure}[h]
  \centering
  \includegraphics[width=\linewidth]{kmeans-silhouette-4.pdf}
  \caption{$k$-means cluster silhouette plot, $k = 4$}
  \label{fig:kmeans-silhouette-4}
\end{figure}

\subsection{Decision trees based on clusters}
% seed = 911
% CLUSTERS: 2
% ================================
% Complexity Parameter: 0.02183406
% 10-fold CV error: 0.1132075
% Root node error: 0.254162
% CLUSTERS: 3
% ================================
% Complexity Parameter: 0.01070664
% 10-fold CV error: 0.190899
% Root node error: 0.518313
% CLUSTERS: 4
% ================================
% Complexity Parameter: 0.01
% 10-fold CV error: 0.2552719
% Root node error: 0.5627081
\begin{table}[h]
  \centering
  \begin{tabular}{l|l|l|l|l|l}
    $k$ & CP\tablefootnote{Complexity Parameter} & CV Xerror\tablefootnote{10-fold cross
    validation} & Root Feature &
    Root Error & Figure \\
    \hline
    2 & 0.0218 & 0.113 & axial $\geq$ 4.5 & 0.254 & Figure~\ref{fig:kmeans-dtree-2} \\
    3 & 0.0107 & 0.191 & pigd $\geq$ 2.5 & 0.518 & Figure~\ref{fig:kmeans-dtree-3} \\
    4 & 0.0100 & 0.255 & pigd $<$ 2.5 & 0.563 & Figure~\ref{fig:kmeans-dtree-4} \\
  \end{tabular}
  \caption{$k$-kmeans decision trees statistics}
  \label{tab:k-means-dtrees}
\end{table}

\begin{figure}[h]
  \centering
  \includegraphics[width=\linewidth]{dtree-kmeans-pruned-unscaled-2.pdf}
  \caption{Decision Tree from $k$-means clustering, 2 clusters}
  \label{fig:kmeans-dtree-2}
\end{figure}

\begin{figure}[h]
  \centering
  \includegraphics[width=\linewidth]{dtree-kmeans-pruned-unscaled-3.pdf}
  \caption{Decision Tree from $k$-means clustering, 3 clusters}
  \label{fig:kmeans-dtree-3}
\end{figure}

\begin{figure}[h]
  \centering
  \includegraphics[width=\linewidth]{dtree-kmeans-pruned-unscaled-4.pdf}
  \caption{Decision Tree from $k$-means clustering, 4 clusters}
  \label{fig:kmeans-dtree-4}
\end{figure}

\subsection{Interpretation of Clusters}

\subsubsection{Cluster summaries}

Available in
Figures~\ref{fig:kmeans-summaries-2},~\ref{fig:kmeans-summaries-3},
and~\ref{fig:kmeans-summaries-4}. Error bar is standard error.

\subsubsection{Interpretation}

$k = 2$ seems too basic. Cluster is organized solely by severity - all
symptoms, including motor and nonmotor, are higher in severity in cluster 1,
and lower in cluster 2. Quite consistently, groups in cluster 1 are generally
of slightly higher age and pd duration.

$k = 3$ seems like a further development of $k = 2$, where clusters are simply
organized by linearly increasing severity.

$k = 4$ is where it gets interesting.

\begin{figure}[h]
  \centering
  \includegraphics[width=\linewidth]{kmeans-summaries-2.pdf}
  \caption{Cluster Summaries, $k = 2$}
  \label{fig:kmeans-summaries-2}
\end{figure}

\begin{figure}[h]
  \centering
  \includegraphics[width=\linewidth]{kmeans-summaries-3.pdf}
  \caption{Cluster Summaries, $k = 3$}
  \label{fig:kmeans-summaries-3}
\end{figure}

\begin{figure}[h]
  \centering
  \includegraphics[width=\linewidth]{kmeans-summaries-4.pdf}
  \caption{Cluster Summaries, $k = 4$}
  \label{fig:kmeans-summaries-4}
\end{figure}

\clearpage
\section{Affinity Propagation}

\subsection{Clustering}

Package \texttt{apcluster} was used. Distance matrix was the negative euclidean
squared distance ($r = 2$).

AP with input preferences minimized ($q = 0$) resulted in 8 clusters.
With the standard median input preferences ($q = 0.5$), algorithm failed to
converge with default parameters. Even setting damping factor to 0.98, maximum
iterations to 10000, and convergence iterations to 1000 failed to converge.
Might need to try a longer run.

\emph{However}, given that input preferences control how many clusters are
found, I don't think it's very useful to have some dozen clusters running
around.

\subsubsection{Silhouette Plots}

Silhouette plot in Figure~\ref{fig:ap-silhouette} looks pretty weak, really.
Tons of overlap between the clusters.

\begin{figure}[h]
  \centering
  \includegraphics[width=\linewidth]{ap-silhouette.pdf}
  \caption{AP silhouette plot, $k = 8$}
  \label{fig:ap-silhouette}
\end{figure}

\clearpage
\section{Hierarchical Clustering}

\subsection{Clustering}

Four dissimilarity methods were used with a euclidean distance matrix.
Dendrograms available in Figure~\ref{fig:hc-dendrograms}

\begin{figure}[h]
  \centering
  \includegraphics[width=\linewidth]{hc-dendrograms.pdf}
  \caption{Dendrograms}
  \label{fig:hc-dendrograms}
\end{figure}

\subsection{Cutting Trees}
\begin{table}[h]
  \centering
  \begin{tabular}{l|l|l|l}
    Method & Condition & n & Figure \\
    \hline
    Complete & $k = 4$ & 4 (665/74/35/7) & ~\ref{fig:hc-summaries-complete-k4} \\
    Complete & \texttt{dynamicTreeCut}\tablefootnote{Package \texttt{dynamicTreeCut} in R (Langfelder P,
  Zhang B, Horvath S (2007)). Hybrid method, minimum cluster selection
parameters} & 11 (7/270/138/83/79/49/46/39/35/35) &
    ~\ref{fig:hc-summaries-complete-dynamic} \\
    Ward & $k = 4$ & 4 (294/236/120/131) & ~\ref{fig:hc-summaries-ward-D-k4} \\
    Ward & $h = 60$ & 6 (97/236/120/197/91/40) &
    ~\ref{fig:hc-summaries-ward-D-h60} \\
  \end{tabular}
  \caption{Clusters from Tree Cutting}
  \label{tab:tree-cutting}
\end{table}

\subsection{Interpretation}

\begin{figure}[h]
  \centering
  \includegraphics[width=\linewidth]{hc-summaries-complete-k4.pdf}
  \caption{Using maximum (complete linkage) dissimilarity, cutting tree for $k = 4$}
  \label{fig:hc-summaries-complete-k4}
\end{figure}

\begin{figure}[h]
  \centering
  \includegraphics[width=\linewidth]{hc-summaries-complete-dynamic.pdf}
  \caption{Using maximum (complete linkage) dissimilarity, cutting tree
    dynamically}
  \label{fig:hc-summaries-complete-dynamic}
\end{figure}

\begin{figure}[h]
  \centering
  \includegraphics[width=\linewidth]{hc-summaries-ward-D-k4.pdf}
  \caption{Using Ward (1963) dissimilarity, cutting tree for $k = 4$}
  \label{fig:hc-summaries-ward-D-k4}
\end{figure}

\begin{figure}[h]
  \centering
  \includegraphics[width=\linewidth]{hc-summaries-ward-D-h60.pdf}
  \caption{Using Ward (1963) dissimilarity, cutting tree at $h = 60$}
  \label{fig:hc-summaries-ward-D-h60}
\end{figure}


\subsection{Interpretation}

Cluster sizes are available in Table~\ref{tab:ap-cluster-sizes}

\begin{table}[h]
  \centering
  \begin{tabular}{l|l}
  Cluster & Size \\
  \hline
  1 & 54 \\
  2 & 68 \\
  3 & 25 \\
  4 & 259 \\
  5 & 102 \\
  6 & 68 \\
  7 & 185 \\
  8 & 20 \\
  \end{tabular}
  \caption{AP Cluster Sizes}
  \label{tab:ap-cluster-sizes}
\end{table}

Boxplot summary of clusters available in Figure~\ref{fig:ap-summaries}.
\textbf{Discussion forthcoming.}

\begin{figure}[h]
  \centering
  \includegraphics[width=\linewidth]{ap-summaries.pdf}
  \caption{AP Boxplot Summaries}
  \label{fig:ap-summaries}
\end{figure}

\section{Biclustering}

Used BCBimax clustering algorithm. Clusters seem quite sparse.

\begin{figure}[h]
  \centering
  \includegraphics[width=\linewidth]{biclust-16.pdf}
  \caption{Biclustering $N = 16$}
  \label{fig:biclust-16}
\end{figure}

\begin{figure}[h]
  \centering
  \includegraphics[width=\linewidth]{biclust-bubbleplot-16.pdf}
  \caption{Bubbleplot $N = 16$}
  \label{fig:bubbleplot-16}
\end{figure}

\section{Subspace clustering}

% TODO: Wat.

\section{Bayesian Networks}

\end{document}
